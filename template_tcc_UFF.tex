% Options for packages loaded elsewhere
\PassOptionsToPackage{unicode}{hyperref}
\PassOptionsToPackage{hyphens}{url}
\PassOptionsToPackage{dvipsnames,svgnames,x11names}{xcolor}
%
\documentclass[
  12pt,
]{article}

\usepackage{amsmath,amssymb}
\usepackage{setspace}
\usepackage{iftex}
\ifPDFTeX
  \usepackage[T1]{fontenc}
  \usepackage[utf8]{inputenc}
  \usepackage{textcomp} % provide euro and other symbols
\else % if luatex or xetex
  \usepackage{unicode-math}
  \defaultfontfeatures{Scale=MatchLowercase}
  \defaultfontfeatures[\rmfamily]{Ligatures=TeX,Scale=1}
\fi
\usepackage{lmodern}
\ifPDFTeX\else  
    % xetex/luatex font selection
    \setmainfont[]{Times New Roman}
\fi
% Use upquote if available, for straight quotes in verbatim environments
\IfFileExists{upquote.sty}{\usepackage{upquote}}{}
\IfFileExists{microtype.sty}{% use microtype if available
  \usepackage[]{microtype}
  \UseMicrotypeSet[protrusion]{basicmath} % disable protrusion for tt fonts
}{}
\makeatletter
\@ifundefined{KOMAClassName}{% if non-KOMA class
  \IfFileExists{parskip.sty}{%
    \usepackage{parskip}
  }{% else
    \setlength{\parindent}{0pt}
    \setlength{\parskip}{6pt plus 2pt minus 1pt}}
}{% if KOMA class
  \KOMAoptions{parskip=half}}
\makeatother
\usepackage{xcolor}
\usepackage[left=3cm, top=3cm, right=2cm, bottom=2cm]{geometry}
\usepackage{listings}
\newcommand{\passthrough}[1]{#1}
\lstset{defaultdialect=[5.3]Lua}
\lstset{defaultdialect=[x86masm]Assembler}
\setlength{\emergencystretch}{3em} % prevent overfull lines
\setcounter{secnumdepth}{5}
% Make \paragraph and \subparagraph free-standing
\makeatletter
\ifx\paragraph\undefined\else
  \let\oldparagraph\paragraph
  \renewcommand{\paragraph}{
    \@ifstar
      \xxxParagraphStar
      \xxxParagraphNoStar
  }
  \newcommand{\xxxParagraphStar}[1]{\oldparagraph*{#1}\mbox{}}
  \newcommand{\xxxParagraphNoStar}[1]{\oldparagraph{#1}\mbox{}}
\fi
\ifx\subparagraph\undefined\else
  \let\oldsubparagraph\subparagraph
  \renewcommand{\subparagraph}{
    \@ifstar
      \xxxSubParagraphStar
      \xxxSubParagraphNoStar
  }
  \newcommand{\xxxSubParagraphStar}[1]{\oldsubparagraph*{#1}\mbox{}}
  \newcommand{\xxxSubParagraphNoStar}[1]{\oldsubparagraph{#1}\mbox{}}
\fi
\makeatother


\providecommand{\tightlist}{%
  \setlength{\itemsep}{0pt}\setlength{\parskip}{0pt}}\usepackage{longtable,booktabs,array}
\usepackage{calc} % for calculating minipage widths
% Correct order of tables after \paragraph or \subparagraph
\usepackage{etoolbox}
\makeatletter
\patchcmd\longtable{\par}{\if@noskipsec\mbox{}\fi\par}{}{}
\makeatother
% Allow footnotes in longtable head/foot
\IfFileExists{footnotehyper.sty}{\usepackage{footnotehyper}}{\usepackage{footnote}}
\makesavenoteenv{longtable}
\usepackage{graphicx}
\makeatletter
\newsavebox\pandoc@box
\newcommand*\pandocbounded[1]{% scales image to fit in text height/width
  \sbox\pandoc@box{#1}%
  \Gscale@div\@tempa{\textheight}{\dimexpr\ht\pandoc@box+\dp\pandoc@box\relax}%
  \Gscale@div\@tempb{\linewidth}{\wd\pandoc@box}%
  \ifdim\@tempb\p@<\@tempa\p@\let\@tempa\@tempb\fi% select the smaller of both
  \ifdim\@tempa\p@<\p@\scalebox{\@tempa}{\usebox\pandoc@box}%
  \else\usebox{\pandoc@box}%
  \fi%
}
% Set default figure placement to htbp
\def\fps@figure{htbp}
\makeatother

\usepackage{tocloft}
\usepackage{titletoc}
\usepackage{sectsty}
\usepackage{fancyhdr}
\usepackage{etoolbox}
\usepackage{float}
\usepackage[hyphens]{url}
\usepackage[breaklinks]{hyperref}
\usepackage{indentfirst}

% Remove recuos e define margens zero
\setlength{\cftbeforetoctitleskip}{0pt}
\setlength{\cftaftertoctitleskip}{1em}
\setlength{\parindent}{1.25cm}

% Define formato dos títulos com espaço para numeração e tamanho 12pt
\titlecontents{section}[2.3em]{\fontsize{12}{14}\selectfont}
{\contentslabel{2.3em}}{}{\titlerule*[1pc]{.}\contentspage}[\vspace{0.5em}]
\sectionfont{\fontsize{12}{14}\selectfont\bfseries}
\subsectionfont{\fontsize{12}{14}\selectfont\bfseries}

\titlecontents{figure}[2.3em]{\fontsize{12}{14}\selectfont}
{\contentslabel{2.3em}}{}{\titlerule*[1pc]{.}\contentspage}[\vspace{0.5em}]

\titlecontents{table}[2.3em]{\fontsize{12}{14}\selectfont}
{\contentslabel{2.3em}}{}{\titlerule*[1pc]{.}\contentspage}[\vspace{0.5em}]

% Formata títulos das listas
\renewcommand{\cfttoctitlefont}{\fontsize{12}{14}\selectfont\textbf}
\renewcommand{\cftloftitlefont}{\fontsize{12}{14}\selectfont\textbf}
\renewcommand{\cftlottitlefont}{\fontsize{12}{14}\selectfont\textbf}

% Lista de equações
\newcommand{\listequationsname}{\fontsize{12}{14}\selectfont\textbf{LISTA DE EQUAÇÕES}}
\newlistof{equations}{equ}{\listequationsname}
\newcommand{\equations}[1]{%
\addcontentsline{equ}{equations}{\protect\numberline{\theequation}#1}\par}

% Configuração da numeração das páginas
\fancypagestyle{plain}{%
    \fancyhf{}%
    \fancyheadoffset[R]{0cm}%
    \renewcommand{\headrulewidth}{0pt}% Remove linha do cabeçalho
    \fancyhead[R]{%
        \ifnum\value{page}<10
            \phantom{\thepage}
        \else
            \thepage
        \fi
    }%
}

\pagestyle{plain}
\setcounter{page}{2}
\makeatletter
\@ifpackageloaded{caption}{}{\usepackage{caption}}
\AtBeginDocument{%
\ifdefined\contentsname
  \renewcommand*\contentsname{Table of contents}
\else
  \newcommand\contentsname{Table of contents}
\fi
\ifdefined\listfigurename
  \renewcommand*\listfigurename{LISTA DE FIGURAS}
\else
  \newcommand\listfigurename{LISTA DE FIGURAS}
\fi
\ifdefined\listtablename
  \renewcommand*\listtablename{LISTA DE TABELAS}
\else
  \newcommand\listtablename{LISTA DE TABELAS}
\fi
\ifdefined\figurename
  \renewcommand*\figurename{Figura}
\else
  \newcommand\figurename{Figura}
\fi
\ifdefined\tablename
  \renewcommand*\tablename{Tabela}
\else
  \newcommand\tablename{Tabela}
\fi
}
\newcommand*\listoflistings\lstlistoflistings
\AtBeginDocument{%
\renewcommand*\lstlistlistingname{List of Listings}
}
\makeatother
\makeatletter
\makeatother
\makeatletter
\@ifpackageloaded{caption}{}{\usepackage{caption}}
\@ifpackageloaded{subcaption}{}{\usepackage{subcaption}}
\makeatother

\usepackage{bookmark}

\IfFileExists{xurl.sty}{\usepackage{xurl}}{} % add URL line breaks if available
\urlstyle{same} % disable monospaced font for URLs
\hypersetup{
  pdftitle={dissertação},
  pdfauthor={Nome do Autor},
  colorlinks=true,
  linkcolor={black},
  filecolor={Maroon},
  citecolor={Blue},
  urlcolor={Blue},
  pdfcreator={LaTeX via pandoc}}


\title{dissertação}
\author{Nome do Autor}
\date{2024-11-27}

\begin{document}

% \pagenumbering{gobble}



\begin{center}
    \includegraphics[angle=0,keepaspectratio,width=3cm]{UFF.png}
    \end{center}
    
    \begin{center}
    \textbf{\fontsize{12}{14}\selectfont 
    UNIVERSIDADE FEDERAL FLUMINENSE\\[0.2cm]
    FACULDADE DE ADMINISTRAÇÃO E CIÊNCIAS CONTÁBEIS\\[0.2cm]
    Programa de Pós-Graduação em Administração\\[0.2cm]
    Mestrado Acadêmico em Administração\\[4.5cm]
    FUNDOS DE INVESTIMENTO IMOBILIÁRIO: UMA ANÁLISE DOS PRINCIPAIS INDICADORES DE PERFORMANCE\\[4cm]
    }
    \end{center}
    
    \begin{center}
    \textbf{MARCUS ANTONIO CARDOSO RAMALHO\\[5.5cm]
    Niterói\\[0.2cm]
    2024
    }
    \end{center}
    \thispagestyle{empty}
    \begin{center}
    
    \textbf{MARCUS ANTONIO CARDOSO RAMALHO\\[1.5cm]
            FUNDOS DE INVESTIMENTO IMOBILIÁRIO: UMA ANÁLISE DOS PRINCIPAIS INDICADORES DE PERFORMANCE\\[5cm]
            }
        
       
        
        \end{center}
    
    
    \begin{quotation}
    \setlength{\leftskip}{7cm}
     \noindent{Dissertação de mestrado apresentada ao Programa de Pós-Graduação em Administração da Faculdade de Administração e Ciências Contábeis da Universidade Federal Fluminense, como requisito  para obtenção do título de Mestre em Administração.\\[1cm]
        Orientador: Prof. Dr. Ariel Levy\\[7.5cm]}
    \end{quotation}
    
    \begin{center}
        Niterói/RJ\\[0.2cm]
        2024 
    \end{center}
    
    \newpage
    
    \begin{center}
        \textbf{\Large Resumo}\\[0.2cm]
    \end{center}
    
    
    
    \begin{flushleft}
        \setlength{\parskip}{1cm} % Espaçamento entre parágrafos
        \linespread{1.5}\selectfont % Espaçamento entre linhas
        \hspace*{0cm}\parbox{16.5cm}{
            Escreva o resumo aqui.
            \linespread{1.5}\selectfont
            \textbf{Palavras-chave:} \textit{Fundos de Investimento Imobiliário; Performance; Variáveis Macroeconômicas; Análise de Séries Temporais.}
        }
    \end{flushleft}
    \newpage
    
    
    
    \begin{center}
        \textbf{\Large Agradecimentos}\\[0.2cm]
    \end{center}
    
    \begin{flushleft}
        \setlength{\parskip}{1cm} % Espaçamento entre parágrafos
        \linespread{1.5}\selectfont % Espaçamento entre linhas
        \hspace*{0cm}\parbox{16.5cm}{
            Escreva a seguir os agradecimentos.
        }
    
    
    \end{flushleft}
    \newpage
    
    
    
    
    
    \thispagestyle{empty}
    
        
    
\renewcommand*\contentsname{SUMÁRIO}
{
\hypersetup{linkcolor=}
\setcounter{tocdepth}{3}
\tableofcontents
}
\listoffigures
\listoftables

\setstretch{1.5}
\newpage
\listofequations
\newpage

\section{Introdução}\label{introduuxe7uxe3o}

\section{Objetivos}\label{objetivos}

\subsection{Gerais}\label{gerais}

\subsection{Específicos}\label{especuxedficos}

\section{Justificativa}\label{justificativa}

\section{Referencial Teórico}\label{referencial-teuxf3rico}

\section{Metodologia}\label{metodologia}

\section{Limitações}\label{limitauxe7uxf5es}

\section{Discussão}\label{discussuxe3o}

\section{Conclusão}\label{conclusuxe3o}

\section{Conclusão}\label{conclusuxe3o-1}

\newpage

\section{Bibliografia}\label{bibliografia}




\end{document}
